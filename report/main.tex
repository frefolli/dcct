\documentclass[a4paper,11pt,oneside, table]{article}
\usepackage[margin=1in]{geometry}
\usepackage{setspace}
\usepackage{imakeidx}
\usepackage{float}
\usepackage{graphicx}
\usepackage{pdfpages}
\usepackage{csquotes}
\usepackage{caption}
\captionsetup[table]{labelfont=it}
\usepackage{pifont}% http://ctan.org/pkg/pifont

\newcommand{\cmark}{\ding{51}}%
\newcommand{\xmark}{\ding{55}}%

\usepackage{listings}
\usepackage{listings-cpp}
\usepackage{algorithm}
\usepackage{algpseudocode}

\newtheorem{nota}{Nota}

\usepackage[italian]{babel}
\usepackage[
  backend=bibtex,
  style=numeric,
  sorting=ydnt
  ]{biblatex}
\addbibresource{refs.bib}
\makeindex

\newcommand{\putimage}[4] {
	\begin{figure}[H]
	    \centering
	    \includegraphics[width={#4}\linewidth]{#1}
	    \caption{#2}\label{#3}
	\end{figure}
}

\newcommand{\putsubimage}[5] {
  \begin{minipage}{{#4}\linewidth}
	    \centering
      \includegraphics[width={#5}\linewidth]{#1}
	    \caption{#2}\label{#3}
	\end{minipage}
}

\newcommand{\putimagecouple}[2] {
  \begin{figure}[!htb]
      \centering
      #1
      \hspace{0.5cm}
      #2
  \end{figure}
}

\newcommand{\putimagequadruple}[4] {
  \begin{figure}[!htb]
      \centering
      #1
      \hspace{0.5cm}
      #2
      \linebreak
      #3
      \hspace{0.5cm}
      #4
  \end{figure}
}

\algnewcommand{\LineComment}[1]{\State \(\triangleright\) #1}

\begin{document}
    \begin{titlepage}
        \noindent
        \begin{minipage}[t]{0.19\textwidth}
            \vspace{-4mm}{\includegraphics[scale=1.15]{logo_unimib.pdf}}
        \end{minipage}
        \begin{minipage}[t]{0.81\textwidth}
        {
                \setstretch{1.42}
                {\textsc{Università degli Studi di Milano - Bicocca}} \\
                \textbf{Scuola di Scienze} \\
                \textbf{Dipartimento di Informatica, Sistemistica e Comunicazione} \\
                \textbf{Corso di laurea magistrale in Informatica} \\
                \par
        }
        \end{minipage}
    	\vspace{40mm}
    	\begin{center}
            {\LARGE{
                    \setstretch{1.2}
                    \textbf{Relazione di Metodi del Calcolo Scientifico - Progetto 2}
                    \par
            }}
        \end{center}
        
        \vspace{50mm}
        
        \vspace{15mm}

        \begin{flushright}
            {\large \textbf{Relazione di:}} \\
            \large{Refolli Francesco}
            \large{865955}
        \end{flushright}
        
        \vspace{40mm}
        \begin{center}
            {\large{\bf Anno Accademico 2023-2024}}
        \end{center}
        \restoregeometry
    \end{titlepage}

    \printindex
    \tableofcontents
    \renewcommand{\baselinestretch}{1.5}

\section{Introduzione}

Lo scopo del progetto \`e implementare le operazioni di Discrete Cosine Transform e confrontare i tempi di esecuzione con quelli di una libreria. Quindi implementare una versione semplificata della compressione di immagini JPEG e visualizzare i risultati della compressione.

\paragraph{Tecnologie utilizzate}

Per l'implementazione si \`e scelto di sviluppare una libreria in C++ con Eigen, una nota libreria Free and Open Source per la gestione di  matrici dense.
Le implementazioni da libreria della DCT che verranno comparate sono:

\begin{itemize}
  \item \textbf{FFTW} \`e una libreria free and open source scritta in linguaggio C riccamente documentata che supporta DCT in modalit\`a fast con la possibilit\`a di creare degli \textit{execution plan} e mantenere informazioni in memoria (\textit{wisdom}) per ottimizzare ulteriormente le operazioni. \`E molto popolare per via della sua velocit\`a, flessibilit\`a e i binding per alcuni dei linguaggi di programmazione maggiormente utilizzati per il calcolo scientifico (Julia e Rust).
  \item \textbf{PocketFFT} \`e una libreria free and open source in linguaggio C/C++ con un'interfaccia non adeguatamente documentata ma molto promettente che viene usata da Scipy (libreria Python per il calcolo scientifico) come implementazione di default dalle ultime release invece della pi\`u nota \textit{fftpack} (libreria Fortran di cui \textit{pocketfft} \`e una reimplementazione).
\end{itemize}

Per verificare la qualit\`a delle implementazioni di funzioni e binding sono stati costruiti dei test basati sull'esecuzione della DCT sulle matrici di esempio presentate nella consegna.

\section{Architettura}

\`E stato definito un \textbf{tratto} per un \textbf{Actuator} che applica le operazioni di DCT e quindi le strutture che implementano le operazioni o i binding alle librerie associate.

\putimage{images/diagram.png}{Diagramma di Strutture e Tratti d'Interesse}{png:diagram_of_structures}{0.99}

Per rendere pi\`u fair la fase di comparazione e pi\`u semplice il testing si \`e deciso di applicare la normalizzazione presentata dalla Consegna anche ai binding. Per facilitare questa operazione tutte le implementazioni della DCT a due dimensioni sono come segue:

\begin{lstlisting}[language=C++]
Eigen::MatrixXd dcct::FFTWActuator::dct2(Eigen::MatrixXd& X) {
  Eigen::MatrixXd Y = dct(X).transpose();
  X = dct(Y).transpose();
  return X;
}
\end{lstlisting}

Sono state inoltre implementate le operazioni inverse \textbf{IDCT}, \textbf{IDCT2} per tutte le implementazioni di Actuator sia perch\`e sono necessarie nella successiva sperimentazione nella compressione delle immagini sia perch\`e permettono di avere ulteriori test che confrontano la correttezza delle implementazioni velocizzando lo sviluppo.

\subsection{Slow}

Si tratta di una implementazione da manuale della Discrete Cosine Trasform che applica anche la normalizzazione.

\begin{lstlisting}[language=C++]
Eigen::MatrixXd dcct::SlowActuator::dct(Eigen::MatrixXd& X) {
  uint32_t N = X.rows(), M = X.cols();
  Eigen::MatrixXd Y(N, M);

  // Coefficienti di normalizzazione
  double_t sqrt_of_two_over_M = std::sqrt(2.0 / (double_t) M);
  double_t one_over_sqrt_of_two = std::sqrt(0.5);
  // Precalcolo di pi/M
  double_t pi_over_M = std::numbers::pi / (double_t) M;

  for (uint32_t i = 0; i < N; ++i) {
    for (uint32_t k = 0; k < M; ++k) {
      double_t sum = 0;
      double_t _k_pi_over_M = pi_over_M * k;
      for (uint32_t j = 0; j < M; j++) {
        sum += std::cos(_k_pi_over_M * (j + 0.5)) * X.coeff(i, j);
      }
      Y.coeffRef(i, k) = sum * sqrt_of_two_over_M;
    }
    Y.coeffRef(i, 0) *= one_over_sqrt_of_two;
  }
  return Y;
}
\end{lstlisting}

\subsection{Fast}

Il nome \textit{Fast} non si riferisce all'applicazione della Fast Fourier Transform ma bens\`i all'utilizzo avanzato di Eigen per ottimizzare i calcoli.
Questa \`e l'implementazione \textit{naive} di bandiera.

\begin{lstlisting}[language=C++]
Eigen::MatrixXd dcct::FastActuator::dct(Eigen::MatrixXd& X) {
  uint32_t N = X.rows(), M = X.cols();
  Eigen::MatrixXd Y(N, M);

  // Coefficienti di normalizzazione
  double_t sqrt_of_two_over_M = std::sqrt(2.0 / (double_t) M);
  double_t one_over_sqrt_of_two = std::sqrt(0.5);
  // Precalcolo di (j + 0.5) * pi / M per ogni j : [0, M)
  double_t pi_over_M = std::numbers::pi / (double_t)M;
  Eigen::VectorXd Z(M);
  for (uint32_t j = 0; j < M; ++j) {
    Z.coeffRef(j) = (j + 0.5) * pi_over_M;
  }

  Eigen::RowVectorXd W;
  X.transposeInPlace();
  for (uint32_t k = 0; k < M; ++k) {
    // W e' un array di coseni delle entrate del vettore Z * k
    W = (Z * k).array().cos();
    // Calcolo la colonna k-esima e gli applico la normalizzazione generica
    Y.col(k) = (W * X).colwise().sum() * sqrt_of_two_over_M;
  }
  // Sistemo la normalizzazione della prima colonna
  Y.col(0) *= one_over_sqrt_of_two;

  return Y;
}
\end{lstlisting}

\subsection{FFTW}

La libreria consiglia di creare un \textbf{execution plan} per migliorare le prestazioni di esecuzione della DCT, e cos\`i \`e stato fatto.

\begin{lstlisting}[language=C++]
Eigen::MatrixXd dcct::FFTWActuator::dct(Eigen::MatrixXd& X) {
  uint32_t N = X.rows(), M = X.cols();
  Eigen::MatrixXd Y(N, M);
  // Coefficienti di normalizzazione
  double_t sqrt_of_one_half_over_N = std::sqrt(0.5 / (double_t) M);
  double_t one_over_sqrt_of_two = std::sqrt(0.5);

  // FFTW_REDFT10 e' la DCT-II
  int dims[1] = {(int)M}; 
  fftw_r2r_kind kinds[1] = {FFTW_REDFT10};
  // Creo un piano con l'istruzione di costruire un output
  // senza modificare l'input, altrimenti l'output non si
  // dovrerebbe nell'input ma quest'ultimo verrebbe utilizzato
  // per precalcolare alcuni fattori.
  fftw_plan plan = fftw_plan_many_r2r(
    1, dims, N, const_cast<double*>(X.data()),
    dims, N, 1, Y.data(),
    dims, N, 1, kinds,
    FFTW_PRESERVE_INPUT);
  fftw_execute_r2r(plan, const_cast<double*>(X.data()), Y.data());
  fftw_destroy_plan(plan);

  // Applico la normalizzazione
  Y *= sqrt_of_one_half_over_N;
  Y.col(0) *= one_over_sqrt_of_two;
  return Y;
}
\end{lstlisting}

\subsection{PocketFFT}

\`E difficile all'inizio prenderci la mano ma una volta che si \`e compreso come ragiona l'API diventa comunque utilizzabile.
Uno dei limiti di questa implementazione \`e che non supporta le matrici non quadrate, possibilmente risolvibile ma in mancanza di una documentazione chiara \`e stato impossibile porre rimedio al problema.

\begin{lstlisting}[language=C++]
#define POCKETFFT_DCT2 2
#define POCKETFFT_SINGLETHREAD 1

Eigen::MatrixXd dcct::PocketFFTActuator::dct(Eigen::MatrixXd& X) {
  uint32_t N = X.rows(), M = X.cols();
  Eigen::MatrixXd Y(N, M);
  // Coefficienti di normalizzazione
  double_t sqrt_of_two_over_M = std::sqrt(0.5 / (double_t) M);
  double_t one_over_sqrt_of_two = std::sqrt(0.5);

  // Cerimony per la dimensione delle matrici
  const pocketfft::shape_t shape = {N, M};
  const pocketfft::stride_t stride_in = {sizeof(double_t),
                                         (long)(sizeof(double_t) * M)};
  const pocketfft::stride_t stride_out = {sizeof(double_t),
                                          (long)(sizeof(double_t) * M)};
  // E l'istruzione su quali assi agire (in questo caso per il lungo)
  const pocketfft::shape_t axes = {1};
  // PocketFFT supporta nativamente il multithread, ma si e' deciso
  // di limitare questo aspetto per fare un confronto piu' equo
  // con il resto delle implementazioni
  pocketfft::dct<double_t>(shape, stride_in,
                           stride_out, axes,
                           POCKETFFT_DCT2, X.data(),
                           Y.data(), 1,
                           false, POCKETFFT_SINGLETHREAD);

  // Applico la normalizzazione
  Y *= sqrt_of_two_over_M;
  Y.col(0) *= one_over_sqrt_of_two;
  return Y;
}
\end{lstlisting}

\subsection{Compressione}

\`E stata implementata una funzione di compressione che prende in input il percorso all'immagine da comprimere, il percorso dove salvare l'output, uno \textbf{specifier} che contiene la configurazione (Actuator da usare, dimensione dei blocchi ... etc) e una coppia di funzioni che possono essere utilizzare dall'esterno per ottenere un feedback di avanzamento della compressione (utile per la GUI che si vedra' in seguito).
Siccome le immagini grayscale bmp sono equivalenti ad un'immagine a 1 canale a 8 bit, si \`e deciso di implementare una funzione che comprime un canale (WIDTH x HEIGHT x 8 bit) da richiamare nella funzione di compressione delle immagini: questo ha permesso di implementare il supporto alle immagini a 3 e 4 colori (RGB e RGBA).
Inoltre durante la divisione dell'immagine in blocchi di dimensione $F$ si \`e deciso di "tagliare" le eccedenze nella misura in cui non vengono processate, ma l'immagine rimane della dimensione originale.
Di seguito un resoconto sotto forma di pseudocodice dell'attivit\`a di compressione.

\begin{algorithm}
  \renewcommand\thealgorithm{}
  \caption{Algoritmo di Compressione}
  \begin{algorithmic}
    \Procedure{CompressChannel}{input\_channel, output\_channel, actuator, F, d}
      \For{$i \gets 0; i < input\_channel.height; i \gets i + F$}
        \LineComment{Si evita di processare le righe in eccesso}
        \If{$i + F > input\_channel.height$}
          \State break
        \EndIf
        \For{$j \gets 0; j < input\_channel.width; j \gets j + F$}
          \LineComment{Si evita di processare le colonne in eccesso}
          \If{$j + F > input\_channel.width$}
            \State break
          \EndIf
          \LineComment{Seleziono il blocco sulla i-esima riga e j-esima colonna di dimensione $F \times F$}
          \State block $\gets$ input\_channel.block(i, j, F, F)
          \State block $\gets$ actuator.dct2(block)
          \State block $\gets$ CutBlock(block, F, d)
          \State block $\gets$ actuator.idct2(block)
          \State block $\gets$ RoundBlock(block, F)
          \LineComment{Scrivo il blocco sulla i-esima riga e j-esima colonna di dimensione $F \times F$}
          \State output\_channel.block(i, j, F, F) $\gets$ block
        \EndFor
      \EndFor
    \EndProcedure
    \Procedure{CompressImage}{input\_filepath, output\_filepath, specifier}
      \State input\_image $\gets$ load\_image(input\_filepath)
      \State output\_image $\gets$ allocate\_as(input\_image)
      \State actuator $\gets$ specifier.actuator()
      \State F $\gets$ specifier.F()
      \State d $\gets$ specifier.d()
      \LineComment{Processo ogni canale separatamente}
      \For{$c \gets 1$ to input\_image.channels}
        \State CompressChannel(input\_image.channel(c), output\_image.channel(c), actuator, F, d)
      \EndFor
      \State write\_image(output\_image, output\_filepath)
    \EndProcedure
  \end{algorithmic}
\end{algorithm}

\subsection{Interfacce}

L'interfaccia da riga di comando permette sia di comprimere immagini, sia di avviare la GUI, sia di eseguire un benchmark parametrizzato da un file \textit{json} (verr\`a utilizzato in seguito).
La configurazione della compressione avviene tramite una stringa \textit{actuator-pattern} che specifica la funzione di compressione (tra le librerie ottimizzate o anche quelle naive), la dimensione del blocco e il treshold di qualit\`a.

\putimage{images/interface-cli.png}{Interfaccia da riga di comando}{png:interface_cli}{0.99}

L'interfaccia GUI permette di selezionare un'immagine dal filesystem, la dimensione del blocco e il treshold di qualit\`a.
Avviata la compressione con il relativo pulsante, la GUI utilizzer\`a la routine di compressione descritta nella sezione precedente scrivendo l'immagine compressa in \textit{/tmp/dcct.output.png} sia per poterla recuperare sia perch\`e \`e pi\`u semplice utilizzare le immagini in Qt se lette dal filesystem.
Finita la compressione (avanzamento segnalato da una barra di progressione) l'immagine originale e la versione compressa sono affiancate e debitamente segnalate.

\putimage{images/interface-gui.png}{Interfaccia GUI in Qt6}{png:interface_gui}{0.99}

\section{Esperimenti}

\subsection{Test di Velocit\`a}

In questa sezione si analizza la velocit\`a delle librerie e dell'implementazioni naive.
Le matrici su cui eseguono la DCT 2D sono $N \times N$ generate randomicamente con $N \in [100, 3000]$.

\putimage{images/actuator-trends.png}{Tempi di Esecuzione in scala semilogaritmica}{png:actuator-trends}{0.99}

Sono riportati nel grafico anche gli andamenti teorici affiancati agli andamenti effettivi delle implementazioni.
Prevedibilmente l'implementazione naive segue un'andamento polinomiale $O(N^3)$ mentre le due librerie sono pi\`u veloci e scalano come $O(N^2 Log N)$.
In particolare si \`e rilevato come \textit{PocketFFT} sia significativamente pi\`u veloce di \textit{FFTW}.

\subsection{Compressione di Immagini}

%Faccio un paio di esempi con matrici in toni di grigio date dal prof
%Quindi dico che ho introdotto il supporto a matrici a colori (perche' sono un perfezionista) separando i canali e comprimendoli singolarmente poi riunendoli (magari una piccola immagine schematica puo' aiutare)
%Quindi faccio un esempio di immagine a colori (magari una ferroviaria) compressa in questo modo
%Eventualmente faccio presente il fenomeno di Gibbs rendendo i blocchi piu' grossi e facendo vedere che con quelli piu' grossi si comprime in modo diverso (piu' veloce per certi versi e spiego il perche')

\printbibliography[title={Bibliografia}]
\end{document}
