\documentclass[a4paper,11pt,oneside, table]{article}
\usepackage[margin=1in]{geometry}
\usepackage{setspace}
\usepackage{imakeidx}
\usepackage{float}
\usepackage{graphicx}
\usepackage{pdfpages}
\usepackage{csquotes}
\usepackage{caption}
\captionsetup[table]{labelfont=it}
\usepackage{pifont}% http://ctan.org/pkg/pifont

\newcommand{\cmark}{\ding{51}}%
\newcommand{\xmark}{\ding{55}}%

\usepackage{listings}
\usepackage{listings-cpp}
\usepackage{algorithm}
\usepackage{algpseudocode}

\newtheorem{nota}{Nota}

\usepackage[italian]{babel}
\usepackage[
  backend=bibtex,
  style=numeric,
  sorting=ydnt
  ]{biblatex}
\addbibresource{refs.bib}
\makeindex

\newcommand{\putimage}[4] {
	\begin{figure}[H]
	    \centering
	    \includegraphics[width={#4}\linewidth]{#1}
	    \caption{#2}\label{#3}
	\end{figure}
}

\newcommand{\putsubimage}[5] {
  \begin{minipage}{{#4}\linewidth}
	    \centering
      \includegraphics[width={#5}\linewidth]{#1}
	    \caption{#2}\label{#3}
	\end{minipage}
}

\newcommand{\putimagecouple}[2] {
  \begin{figure}[!htb]
      \centering
      #1
      \hspace{0.5cm}
      #2
  \end{figure}
}

\newcommand{\putimagequadruple}[4] {
  \begin{figure}[!htb]
      \centering
      #1
      \hspace{0.5cm}
      #2
      \linebreak
      #3
      \hspace{0.5cm}
      #4
  \end{figure}
}

\begin{document}
    \begin{titlepage}
        \noindent
        \begin{minipage}[t]{0.19\textwidth}
            \vspace{-4mm}{\includegraphics[scale=1.15]{logo_unimib.pdf}}
        \end{minipage}
        \begin{minipage}[t]{0.81\textwidth}
        {
                \setstretch{1.42}
                {\textsc{Università degli Studi di Milano - Bicocca}} \\
                \textbf{Scuola di Scienze} \\
                \textbf{Dipartimento di Informatica, Sistemistica e Comunicazione} \\
                \textbf{Corso di laurea magistrale in Informatica} \\
                \par
        }
        \end{minipage}
    	\vspace{40mm}
    	\begin{center}
            {\LARGE{
                    \setstretch{1.2}
                    \textbf{Relazione di Metodi del Calcolo Scientifico - Progetto 2}
                    \par
            }}
        \end{center}
        
        \vspace{50mm}
        
        \vspace{15mm}

        \begin{flushright}
            {\large \textbf{Relazione di:}} \\
            \large{Refolli Francesco}
            \large{865955}
        \end{flushright}
        
        \vspace{40mm}
        \begin{center}
            {\large{\bf Anno Accademico 2023-2024}}
        \end{center}
        \restoregeometry
    \end{titlepage}

    \printindex
    \tableofcontents
    \renewcommand{\baselinestretch}{1.5}

\section{Introduzione}

%Introduzione con la consegna e le tecnologie utilizzate (linguaggio e librerie).
%FFTW e' una libreria free and open source con una interfaccia meh (maggiormente utilizzata sotto binding in Julia, cita studio fatto in Julia in Esperimenti).
%PocketFFT e' una libreria free and open source con una interfaccia non documentata ma molto promettente che viene usata da scipy (cita studio fatto in Python in Esperimenti).

\section{Architettura}

%Sezione con la struttura del codice del progetto con
%- un diagramma d2 TALA
%- con la spiegazione del tratto Actuator
%- gli snippet di codice delle DCT implementate dalle singole strutture
%- gli snippet di codice delle IDCT implementate dalle singole strutture
%- uno snippet per mostrare che DCT2,IDCT2 sono "naive"
%Gli snippet sarebbe meglio a questo punto farli in C++ con listings per mostrare i binding e cose cosi', pero' magari diminuendo il carattere dello stile

\section{Esperimenti}

%brevissima introduzione agli esperimenti che sono stati condotti

\subsection{Test di Velocit\`a}

%in questa sezione analizziamo la velocita' delle librerie
%in particolare devo dire come ho trovato le matrici (le ho create io)
%e mostrare il grafo semilogy dei tempi di esecuzione
%quindi dire che si il mio e' N^3 e il loro e' N^2 LogN
%PocketFFT e' piu' veloce di FFTW anche se ha problemi con matrici non quadrate (ai fini della compressione di immagini non cambia un casu)

\subsection{Compressione di Immagini}

%Faccio un paio di esempi con matrici in toni di grigio date dal prof
%Quindi dico che ho introdotto il supporto a matrici a colori (perche' sono un perfezionista) separando i canali e comprimendoli singolarmente poi riunendoli (magari una piccola immagine schematica puo' aiutare)
%Quindi faccio un esempio di immagine a colori (magari una ferroviaria) compressa in questo modo
%Eventualmente faccio presente il fenomeno di Gibbs rendendo i blocchi piu' grossi e facendo vedere che con quelli piu' grossi si comprime in modo diverso (piu' veloce per certi versi e spiego il perche')

\printbibliography[title={Bibliografia}]
\end{document}
