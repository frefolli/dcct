\documentclass[a4paper,11pt,oneside, table]{article}
\usepackage[margin=1in]{geometry}
\usepackage{setspace}
\usepackage{imakeidx}
\usepackage{float}
\usepackage{graphicx}
\usepackage{pdfpages}
\usepackage{csquotes}
\usepackage{caption}
\captionsetup[table]{labelfont=it}
\usepackage{pifont}% http://ctan.org/pkg/pifont

\newcommand{\cmark}{\ding{51}}%
\newcommand{\xmark}{\ding{55}}%

\usepackage{listings}
\usepackage{listings-cpp}
\usepackage{algorithm}
\usepackage{algpseudocode}

\newtheorem{nota}{Nota}

\usepackage[italian]{babel}
\usepackage[
  backend=bibtex,
  style=numeric,
  sorting=ydnt
  ]{biblatex}
\addbibresource{refs.bib}
\makeindex

\newcommand{\putimage}[4] {
	\begin{figure}[H]
	    \centering
	    \includegraphics[width={#4}\linewidth]{#1}
	    \caption{#2}\label{#3}
	\end{figure}
}

\newcommand{\putsubimage}[5] {
  \begin{minipage}{{#4}\linewidth}
	    \centering
      \includegraphics[width={#5}\linewidth]{#1}
	    \caption{#2}\label{#3}
	\end{minipage}
}

\newcommand{\putimagecouple}[2] {
  \begin{figure}[!htb]
      \centering
      #1
      \hspace{0.5cm}
      #2
  \end{figure}
}

\newcommand{\putimagequadruple}[4] {
  \begin{figure}[!htb]
      \centering
      #1
      \hspace{0.5cm}
      #2
      \linebreak
      #3
      \hspace{0.5cm}
      #4
  \end{figure}
}

\begin{document}
    \begin{titlepage}
        \noindent
        \begin{minipage}[t]{0.19\textwidth}
            \vspace{-4mm}{\includegraphics[scale=1.15]{logo_unimib.pdf}}
        \end{minipage}
        \begin{minipage}[t]{0.81\textwidth}
        {
                \setstretch{1.42}
                {\textsc{Università degli Studi di Milano - Bicocca}} \\
                \textbf{Scuola di Scienze} \\
                \textbf{Dipartimento di Informatica, Sistemistica e Comunicazione} \\
                \textbf{Corso di laurea magistrale in Informatica} \\
                \par
        }
        \end{minipage}
    	\vspace{40mm}
    	\begin{center}
            {\LARGE{
                    \setstretch{1.2}
                    \textbf{Relazione di Metodi del Calcolo Scientifico - Progetto 2}
                    \par
            }}
        \end{center}
        
        \vspace{50mm}
        
        \vspace{15mm}

        \begin{flushright}
            {\large \textbf{Relazione di:}} \\
            \large{Refolli Francesco}
            \large{865955}
        \end{flushright}
        
        \vspace{40mm}
        \begin{center}
            {\large{\bf Anno Accademico 2023-2024}}
        \end{center}
        \restoregeometry
    \end{titlepage}

    \printindex
    \tableofcontents
    \renewcommand{\baselinestretch}{1.5}

\section{Introduzione}

Lo scopo del progetto \`e implementare le operazioni di Discrete Cosine Transform e confrontare i tempi di esecuzione con quelli di una libreria. Quindi implementare una versione semplificata della compressione di immagini JPEG e visualizzare i risultati della compressione.

\paragraph{Tecnologie utilizzate}

Per l'implementazione si \`e scelto di sviluppare una libreria in C++ con Eigen, una nota libreria Free and Open Source per la gestione di  matrici dense.
Le implementazioni da libreria della DCT che verranno comparate sono:

\begin{itemize}
  \item \textbf{FFTW} \`e una libreria free and open source scritta in linguaggio C riccamente documentata che supporta DCT in modalit\`a fast con la possibilit\`a di creare degli \textit{execution plan} e mantenere informazioni in memoria (\textit{wisdom}) per ottimizzare ulteriormente le operazioni. \`E molto popolare per via della sua velocit\`a, flessibilit\`a e i binding per alcuni dei linguaggi di programmazione maggiormente utilizzati per il calcolo scientifico (Julia e Rust).
  \item \textbf{PocketFFT} \`e una libreria free and open source in linguaggio C/C++ con un'interfaccia non adeguatamente documentata ma molto promettente che viene usata da Scipy (libreria Python per il calcolo scientifico) come implementazione di default dalle ultime release invece della pi\`u nota \textit{fftpack} (libreria Fortran di cui \textit{pocketfft} \`e una reimplementazione).
\end{itemize}

Per verificare la qualit\`a delle implementazioni di funzioni e binding sono stati costruiti dei test basati sull'esecuzione della DCT sulle matrici di esempio presentate nella consegna.

\section{Architettura}

\`E stato definito un \textbf{tratto} per un \textbf{Actuator} che applica le operazioni di DCT e quindi le strutture che implementano le operazioni o i binding alle librerie associate.

\putimage{images/diagram.png}{Diagramma di Strutture e Tratti d'Interesse}{png:diagram_of_structures}{0.99}

Per rendere pi\`u fair la fase di comparazione e pi\`u semplice il testing si \`e deciso di applicare la normalizzazione presentata dalla Consegna anche ai binding. Per facilitare questa operazione tutte le implementazioni della DCT a due dimensioni sono come segue:

\begin{lstlisting}[language=C++]
Eigen::MatrixXd dcct::FFTWActuator::dct2(Eigen::MatrixXd& X) {
  Eigen::MatrixXd Y = dct(X).transpose();
  X = dct(Y).transpose();
  return X;
}
\end{lstlisting}

Sono state inoltre implementate le operazioni inverse \textbf{IDCT}, \textbf{IDCT2} per tutte le implementazioni di Actuator sia perch\`e sono necessarie nella successiva sperimentazione nella compressione delle immagini sia perch\`e permettono di avere ulteriori test che confrontano la correttezza delle implementazioni velocizzando lo sviluppo.

\subsection{Slow}

Si tratta di una implementazione da manuale della Discrete Cosine Trasform che applica anche la normalizzazione.

\begin{lstlisting}[language=C++]
Eigen::MatrixXd dcct::SlowActuator::dct(Eigen::MatrixXd& X) {
  uint32_t N = X.rows(), M = X.cols();
  Eigen::MatrixXd Y(N, M);

  double_t pi_over_M = std::numbers::pi / (double_t) M;
  double_t sqrt_of_two_over_M = std::sqrt(2.0 / (double_t) M);
  double_t one_over_sqrt_of_two = std::sqrt(0.5);

  for (uint32_t i = 0; i < N; ++i) {
    for (uint32_t k = 0; k < M; ++k) {
      double_t sum = 0;
      double_t _k_pi_over_M = pi_over_M * k;
      for (uint32_t j = 0; j < M; j++) {
        sum += std::cos(_k_pi_over_M * (j + 0.5)) * X.coeff(i, j);
      }
      Y.coeffRef(i, k) = sum * sqrt_of_two_over_M;
    }
    Y.coeffRef(i, 0) *= one_over_sqrt_of_two;
  }
  return Y;
}
\end{lstlisting}

\subsection{Fast}

Il nome \textit{Fast} non si riferisce all'applicazione della Fast Fourier Transform ma bens\`i all'utilizzo avanzato di Eigen per ottimizzare i calcoli.
Questa \`e l'implementazione \textit{naive} di bandiera.

\begin{lstlisting}[language=C++]
Eigen::MatrixXd dcct::FastActuator::dct(Eigen::MatrixXd& X) {
  uint32_t N = X.rows(), M = X.cols();
  Eigen::MatrixXd Y(N, M);

  double_t pi_over_M = std::numbers::pi / (double_t)M;
  double_t sqrt_of_two_over_M = std::sqrt(2.0 / (double_t) M);
  double_t one_over_sqrt_of_two = std::sqrt(0.5);
  Eigen::VectorXd Z(M);
  for (uint32_t j = 0; j < M; ++j) {
    Z.coeffRef(j) = (j + 0.5) * pi_over_M;
  }

  Eigen::RowVectorXd W;
  X.transposeInPlace();
  for (uint32_t k = 0; k < M; ++k) {
    W = (Z * k).array().cos();
    Y.col(k) = (W * X).colwise().sum() * sqrt_of_two_over_M;
  }
  Y.col(0) *= one_over_sqrt_of_two;

  return Y;
}
\end{lstlisting}

\subsection{FFTW}

\begin{lstlisting}[language=C++]
Eigen::MatrixXd dcct::FFTWActuator::dct(Eigen::MatrixXd& X) {
  uint32_t N = X.rows(), M = X.cols();
  Eigen::MatrixXd Y(N, M);
  double_t sqrt_of_one_half_over_N = std::sqrt(0.5 / (double_t) M);
  double_t one_over_sqrt_of_two = std::sqrt(0.5);

  int dims[1] = {(int)M}; 
  fftw_r2r_kind kinds[1] = {FFTW_REDFT10};
  fftw_plan plan = fftw_plan_many_r2r(
    1, dims, N, const_cast<double*>(X.data()),
    dims, N, 1, Y.data(),
    dims, N, 1, kinds,
    FFTW_PRESERVE_INPUT);
  fftw_execute_r2r(plan, const_cast<double*>(X.data()), Y.data());
  fftw_destroy_plan(plan);

  Y *= sqrt_of_one_half_over_N;
  Y.col(0) *= one_over_sqrt_of_two;
  return Y;
}
\end{lstlisting}

\subsection{PocketFFT}

\begin{lstlisting}[language=C++]
#define POCKETFFT_DCT2 2
#define POCKETFFT_SINGLETHREAD 1

Eigen::MatrixXd dcct::PocketFFTActuator::dct(Eigen::MatrixXd& X) {
  uint32_t N = X.rows(), M = X.cols();
  Eigen::MatrixXd Y(N, M);
  double_t sqrt_of_two_over_M = std::sqrt(0.5 / (double_t) M);
  double_t one_over_sqrt_of_two = std::sqrt(0.5);

  const pocketfft::shape_t shape = {N, M};
  const pocketfft::stride_t stride_in = {sizeof(double_t),
                                         (long)(sizeof(double_t) * M)};
  const pocketfft::stride_t stride_out = {sizeof(double_t),
                                          (long)(sizeof(double_t) * M)};
  const pocketfft::shape_t axes = {1};
  pocketfft::dct<double_t>(shape, stride_in,
                           stride_out, axes,
                           POCKETFFT_DCT2, X.data(),
                           Y.data(), 1,
                           false, POCKETFFT_SINGLETHREAD);

  Y *= sqrt_of_two_over_M;
  Y.col(0) *= one_over_sqrt_of_two;
  return Y;
}
\end{lstlisting}

%Sezione con la struttura del codice del progetto con
%-OK un diagramma d2 TALA
%-OK con la spiegazione del tratto Actuator
%- gli snippet di codice delle DCT implementate dalle singole strutture
%- gli snippet di codice delle IDCT implementate dalle singole strutture
%- uno snippet per mostrare che DCT2,IDCT2 sono "naive"
%Gli snippet sarebbe meglio a questo punto farli in C++ con listings per mostrare i binding e cose cosi', pero' magari diminuendo il carattere dello stile

\section{Esperimenti}

%brevissima introduzione agli esperimenti che sono stati condotti

\subsection{Test di Velocit\`a}

%in questa sezione analizziamo la velocita' delle librerie
%in particolare devo dire come ho trovato le matrici (le ho create io)
%e mostrare il grafo semilogy dei tempi di esecuzione
%quindi dire che si il mio e' N^3 e il loro e' N^2 LogN
%PocketFFT e' piu' veloce di FFTW anche se ha problemi con matrici non quadrate (ai fini della compressione di immagini non cambia un casu)

\subsection{Compressione di Immagini}

%Faccio un paio di esempi con matrici in toni di grigio date dal prof
%Quindi dico che ho introdotto il supporto a matrici a colori (perche' sono un perfezionista) separando i canali e comprimendoli singolarmente poi riunendoli (magari una piccola immagine schematica puo' aiutare)
%Quindi faccio un esempio di immagine a colori (magari una ferroviaria) compressa in questo modo
%Eventualmente faccio presente il fenomeno di Gibbs rendendo i blocchi piu' grossi e facendo vedere che con quelli piu' grossi si comprime in modo diverso (piu' veloce per certi versi e spiego il perche')

\printbibliography[title={Bibliografia}]
\end{document}
